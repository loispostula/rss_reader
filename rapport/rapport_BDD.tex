\documentclass[a4paper,10pt]{article}
\usepackage[utf8]{inputenc}
 \usepackage{latexsym}

\begin{document}

\subsection{Requête 1}
\subsubsection{SQL}
\begin{verbatim}
SELECT u.email, u.password, u.nickname, u.city, u.country, u.avatar, u.biography, u.joinedDate 
FROM user u
LEFT OUTER JOIN friendship f
ON
(
f.user1_email = u.email 
OR 
f.user2_email = u.email
)
GROUP BY u.email
HAVING COUNT(u.email) < 3
\end{verbatim}
\subsection{Algèbre relationnel}
$$Request \leftarrow user\sqsupset\Join_{email=user1\_email \vee email=user2\_email}( friendship)$$
$$Request\_Accepted \leftarrow\ \sigma_{accepted=TRUE}\ (Request)$$
$$Result \leftarrow \pi_* (mail\ \sigma_{COUNT(email)\ <\ 3}\ (Request_Accepted) )$$
\subsubsection{Calcul relationnel tuple}


\subsection{Requête 2}
\subsubsection{SQL}
\begin{verbatim}


\end{verbatim}
\subsection{Algèbre relationnel}

\subsubsection{Calcul relationnel tuple}

\subsection{Requête 3}
\subsubsection{SQL}
\begin{verbatim}


\end{verbatim}
\subsection{Algèbre relationnel}

\subsubsection{Calcul relationnel tuple}

\subsection{Requête 4}
\subsubsection{SQL}
\begin{verbatim}


\end{verbatim}
\subsection{Algèbre relationnel}

\subsubsection{Calcul relationnel tuple}

\subsection{Requête 5}
\subsubsection{SQL}
\begin{verbatim}


\end{verbatim}
\subsection{Algèbre relationnel}

\subsubsection{Calcul relationnel tuple}

\subsection{Requête 6}
\subsubsection{SQL}
\begin{verbatim}


\end{verbatim}
\subsection{Algèbre relationnel}

\subsubsection{Calcul relationnel tuple}

\subsection{Requête 2}
\subsubsection{SQL}
\begin{verbatim}




      
\end{document}
