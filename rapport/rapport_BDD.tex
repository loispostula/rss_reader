\documentclass[a4paper,10pt]{article}
\usepackage[utf8]{inputenc}
 \usepackage{latexsym}
\usepackage{listings}
\usepackage[usenames,dvipsnames]{color}

\usepackage{tikz}


\lstset{
  breaklines=true,                                     % line wrapping on
  language=SQL,
  frame=ltrb,
  framesep=5pt,
  basicstyle=\normalsize,
  keywordstyle=\ttfamily\color{ForestGreen},
  identifierstyle=\ttfamily\color{Cerulean}\bfseries,
  identifierstyle=\ttfamily\color{Cerulean}\bfseries,
  commentstyle=\color{Brown},
  stringstyle=\ttfamily,
  showstringspaces=true,
  numbers=left
}


\begin{document}
\section{Requêtes}
\subsection{Requête 1}
Tous les utilisateurs qui ont au plus 2 amis.
\subsubsection{SQL}
\begin{lstlisting}
SELECT * FROM user u
LEFT OUTER JOIN friendship f
ON
(
f.user1_email = u.email
OR
f.user2_email = u.email
)
GROUP BY u.email
HAVING COUNT(u.email) < 3
\end{lstlisting}
\subsubsection{Algèbre relationnel tuple}
$$Request \leftarrow user\sqsupset\Join_{email=user1\_email \vee email=user2\_email}( friendship)$$
$$Request\_Accepted \leftarrow\ \sigma_{accepted=TRUE}\ (Request)$$
$$Result \leftarrow \pi_* (email\ \sigma_{COUNT(email)\ <\ 3}\ (Request_Accepted) )$$
\subsubsection{Calcul relationnel tuple}
\clearpage
\subsection{Requête 2}
La liste des flux auxquels a souscrit au moins un utilisateur qui a souscrit à au moins deux flux auxquel X
a souscrit.
\subsubsection{SQL}
\begin{lstlisting}
SELECT * FROM feed f
INNER JOIN feedsubscription c ON c.feed_url = f.url
INNER JOIN 
  (
  SELECT b.user_email FROM feedsubscription b 
  INNER JOIN 
    (SELECT feed_url FROM feedsubscription WHERE user_email = "X.email") a
  ON a.feed_url = b.feed_url 
  GROUP BY b.user_email
  HAVING COUNT(b.user_email) > 1
  ) d
ON d.user_email = c.user_email
GROUP BY c.feed_url
\end{lstlisting}
\subsubsection{Algèbre relationnel}

$$a\ \leftarrow\ \pi_{feed\_url}\ (\sigma_{user\_email=<user>.email}\ (feedsubscription)$$
$$b\ \leftarrow\ \pi_{user\_email, feed\_url}\ (feedsubscription)$$
$$b\_a\_join\ \leftarrow\ b \Join_{a.feed\_url=b.feed\_url}\ (a)$$
$$d\ \leftarrow\ \pi_{b.user\_email}\ (b.user\_email\ \sigma_{COUNT(b.user\_email)\ >\ 1} (b\_a\_join))$$
$$c\ \leftarrow\ feedsubscription \Join_{feed\_url=url}\ (f)$$
$$c\_d\_join\ \leftarrow\ c \Join_{c.user\_email=d.user\_email} (d)$$
$$Result\ \leftarrow\ \pi_*\ (c.feed\_url\ c\_d\_join)$$

\subsubsection{Calcul relationnel tuple}
\clearpage
\subsection{Requête 3}
La liste des flux auxquels X a souscrit, auxquels aucun de ses amis n’a souscrit et duquel il n’a partagé
aucune publication.
\subsubsection{SQL}

\begin{lstlisting}


\end{lstlisting}
\subsubsection{Algèbre relationnel}

$$Friend\_Feed \leftarrow feedsubscription \Join_{((user1\_email=email \wedge user2\_email={user.email}) \vee \\ (user2\_email=email \wedge user2\_ema1l=[user.email])) \wedge accepted=TRUE}\ (friendship)$$
$$Friend\_Subscribed \leftarrow \pi_{feed\_url} \sigma_{feed\_url=url}\ (Friend\_Feedsubscriptions)$$

$$Shared\_Feed \leftarrow contain \Join_{publication\_url\_1=publication\_url\_2 \wedge feed\_url\_2 = "feed://"{user.email}}\ (contain)$$
$$Shared\_Feed \leftarrow \pi_{feed_url} \sigma_{feed\_url\_2=url}\ (Shared\_Feed)$$

$$User\_Feed \leftarrow feed \Join_{feed\_url=url \wedge user\_ema1l={user.email}}\ (feedsubscription)$$

$$Result \leftarrow (User\_Feed - Shared\_Feed) - Friend\_Subscribed$$

\subsubsection{Calcul relationnel tuple}
\clearpage
\subsection{Requête 4}
La liste des utilisateurs qui ont partagé au moins 3 publications que X a partagé.
\subsubsection{SQL}
\begin{lstlisting}


\end{lstlisting}
\subsubsection{Algèbre relationnel tuple}

\subsubsection{Calcul relationnel tuple}
\clearpage
\subsection{Requête 5}
La liste des flux auquel un utilisateur est inscrit avec le nombre de publications lues, le nombre de publications
partagées, le pourcentage de ces dernières par rapport aux premières, cela pour les 30 derniers jours et ordonnée
par le nombre de publications partagées.
\subsubsection{SQL}
\begin{lstlisting}


\end{lstlisting}
\subsection{Requête 6}
La liste des amis d’un utilisateur avec pour chacun le nombre de publications lues par jour et le nombre
d’amis, ordonnée par la moyenne des lectures par jour depuis la date d’inscription de cet ami

\subsubsection{SQL}
\begin{lstlisting}


\end{lstlisting}
\end{document}
